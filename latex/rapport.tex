\documentclass[a4paper,12pt]{report}

\usepackage{rapportutc}
\usepackage{setspace}
\usepackage[final]{pdfpages}

\title{\textbf{Rapport de période d'apprentissage} \\Troisième année}
\author{Thomas \bsc{kieffer}}
\date{\today}

\uv{PAE03}
\semestre{A15 - P16}
\branche{Génie Informatique}
\filiere{Ingénierie des}
\specialite{Systèmes d'Information}

\suiveurutc{Vincent \bsc{fremont}}
\suiveurets{Guillaume \bsc{gesquière}}
\entreprise{Vallourec Tubes France}
\logoets{\includegraphics[width=6cm]{img/logoets.jpg}}
\lieustage{CTIV Saint-Saulve, France}


\begin{document}
\stagepdtitre
\restoregeometry
\onehalfspacing
\tableofcontents

\chapter{Remerciements}
\paragraph{}
Avant de commencer ce rapport je tiens à remercier les personnes avec qui j'ai travaillé au cours de cette dernière année : Romain, Rémi, Jérémy, Aurélie, Klaus et Andreas de l'équipe \textit{Lan \& Cabling}, Frédéric et Benjamin de l'équipe \textit{ToIP, Wan \& Global Services}, Stéphane, Jérôme et Maxime de l'équipe \textit{Internet Security}, pour toutes le connaissances qu'ils m'ont apportées. Et bien évidemment Guillaume pour sa disponibilité et son écoute.

\paragraph{}
Merci aussi à tous mes collègues du CTIV, notamment les équipes \textit{Server}, \textit{Workstation}, \textit{Windows}, \textit{Linux}, pour leurs enseignements.

\paragraph{}
Merci également aux responsables apprentissage de l'UTC  ainsi qu'à mon tuteur, M. Fremont, qui on su être disponibles et de bon conseil.

\chapter{Introduction}
\section{Objectifs du rapport}
Ce rapport représente la clôture de ma troisième (et dernière) année d'apprentissage, une nouvelle année riche en changements et en enseignements variées. Je présenterai les différentes activités que j'ai été amené à réaliser au cours de cette période en expliquant leur intérêt dans ma formation. Ce document va également me permettre de prendre le recul nécessaire à l'analyse des ces trois années passées au sein de Vallourec, de faire une mise au point sur mes acquis, mes compétences et ma capacité à exercer le métier auquel j'ai été formé. Je reviendrai également sur ma vision du métier d'ingénieur et mon aptitude à l'exercer.

\section{Contexte d'entreprise}
Cette section a pour objectif de remettre brièvement le lecteur dans le contexte de Vallourec afin de mieux comprendre mon environnement de travail. Ce contexte ayant peu évolué en comparaison de l'année précédente, les idées sont très semblables à celles exprimées en introduction de mon rapport de l'année précédente.

\paragraph{}
Le groupe Vallourec fournit des solutions tubulaires pour de nombreuses applications industrielles. 
Les deux tiers du chiffre d'affaires annuel de Vallourec proviennent du marché du pétrole et du gaz. Par conséquent, la santé de l'entreprise en est très dépendante. L pris très faibles du baril de pétrole au cours de cette année ont fortement impacté les clients de Vallourec qui ont ralenti leur rythme de commandes de tubes. L'activité des usines a donc continué à ralentir, ce qui a conduit à des périodes de chômage partiel voire à der fermetures de lignes de production. Comme ultime conséquence le plan Vallourec a annoncé début 2016 un plan social visant à réduire d'au moins un tiers le nombre de postes en France. Les fonctions support telles que l'informatique on également souffert du ralentissement de l'activité des usines : celles-ci n'ayant plus moyen de financer les projets d'évolution, ceux-ci se retrouvent bloqués.

\section{L'équipe, rôle et enjeux}
Mon poste de travail est basé au Centre de Traitement de l'Information de Vallourec (CTIV), lui-même localisé à Saint-Saulve près de Valenciennes. Le CTIV est le cœur informatique de Vallourec sur l'Europe, l'Afrique et le Moyen-Orient (\textit{EMEA}), c'est ici que sont prises les décisions les plus importantes concernant l'informatique, et que sont gérés les incidents, les problèmes et les évolutions d'architecture.
\paragraph{}
Au sein du département \textit{Infrastructure}, j'appartiens à l'équipe réseau du CTIV, cette équipe est elle même subdivisée en trois sous-équipes :
\begin{itemize}
\item L'équipe Internet Sécurité qui gère les communications entre le réseau interne Vallourec et l'extérieur, ce qui inclut les accès internet et les systèmes de protection tels que les pare-feu.
\item L'équipe ToIP, WAN \& Global Services est responsable de l'infrastructure de téléphonie sur IP ainsi que des connexions inter-sites du réseau Vallourec.
\item L'équipe LAN \& Cabling, à laquelle j'appartiens, est en charge des connexion réseau intra-sites, de leur maintenance et de leur évolution.
\end{itemize}

\paragraph{}
De même que la seconde année, ma mission en entreprise a été confondue à celle de l'équipe réseau, j'ai été amené à travailler sur tout les missions qui ont été confiées aux membres de l'équipe. Le réseau informatique est essentiel pour l'entreprise, toutes les communications entre les employés y transitent. Il est d'autant plus important au CTIV puis qu'il permet la communication entre les serveurs, les systèmes de sécurité et leurs administrateurs.

\section{Bilan de l'année 2}
Au terme de la seconde année d'apprentissage, je tirai un bilan très positif de ma progression : j'avais acquis de nombreuses connaissances techniques qui me permettaient de prendre en charge la grande majorité des incidents qui survenaient soumis à l'équipe ce qui m'a permis d'être beaucoup plus autonome. Au delà du plan technique, j'avais également acquis une meilleur aisance relationnelle qui me permettait d'échanger plus facilement et efficacement avec mes collègues.

\paragraph{}
Ma progression à été très importante au cours de cette année, ce dont je suis assez fier. La différence avec la première année est importante, et j'espère qu'il en sera de même pour la troisième année. Troisième année au cours de laquelle j'espère également avoir la possibilité de m'investir encore plus dans la gestion de projet ce qui me permettra d'améliorer ma capacité de décision ainsi que mes connaissances techniques évidemment.

\section{Objectifs et souhaits} %de moi perso
Au début de cette troisième année, j'espérais progresser aussi vite qu'au cours de la seconde année. Je souhaitais augmenter encore mes compétences techniques mais également monter en compétences sur un aspect plus organisationnel, gestion de tâches/projets et prendre plus de responsabilités que je n'en avais lors de l'année précédente.\\
Un des points qui me tenait à cœur au début de cette troisième année était de m'éloigner progressivement des tâches de maintenance opérationnelle pour me consacrer à des projets d'évolutions du parc. Je souhaitais également avoir la possibilité de prendre en charge un projet afin d'en découvrir les aspects et les méthodes de gestion.

%AJOUTER DES CHOSES ICI

\chapter{Activités réalisées}
Cette partie va me permette de décrire les activités réalisées au cours de cette dernière année. Pour chacune d'entre elles, en plus d'expliquer en qui elles ont consisté, je tenterai d'analyser l'intérêt que ces tâches et projets ont eu pour ma formation, les différentes enseignements que j'en ai tiré.
\paragraph{}
Au cours de cette année, j'ai été amené à contribuer à de nombreux projets et tâches. Néanmoins, au delà de la maintenance opérationnelle effectuée au quotidien, la grande majorité de ces tâches s'inscrivaient dans des grandes catégories : Sécurisation du réseau, supervision \& gestion des équipements, architecture WiFi. Du fait que ces sujets m'intéressaient particulièrement, je m'y suis auto-formé lors des périodes mois chargées ce qui m'a permis d'acquérir des compétences très solides sur ces sujets et ainsi de devenir très performant. C'était donc les sujets qui m'étaient confiés en priorité.

\section{Maintenance opérationnelle}
De même que l'année précédente, la maintenance opérationnelle (appelée \textit{run}) a occupé une partie importante de mon temps au cours de cette année. Pour rappel, le \textit{run} regroupe un grand nombre de tâches telles que la configuration d'équipements, le déplacement sur site pour changer un appareil défectueux, l'application de mises à jour sur les équipements en production, la résolution d'incidents, etc ... En résumé, il s'agit de toutes les actions que l'équipe doit entreprendre pour assurer le fonctionnement du réseau informatique au quotidien et pour que les utilisateurs puissent travailler. Comme les années précédentes je ne pourrai pas décrire toutes les tâches réalisées dans le cadre du \textit{run}. Dans les paragraphes suivants, je présenterai les tâches et aspects les plus intéressants du \textit{run} sur lesquels j'ai travaillé au cours de cette année.

\subsection{Support client}
Le support client est le caractère principal du \textit{run}. En ce qui me concerne, cet aspect a peu changé par rapport aux années précédentes. J'ai été en contact avec les équipes informatiques locales (sur les sites Vallourec distants) qui sont les points de remontées des incidents qui surviennent sur leurs sites. Ceux-ci nous contactent alors pour soumettre ces incidents afin de mettre en place une solution.
\paragraph{}
J'ai néanmoins acquis une aisance supplémentaire par rapport à la prise ne charge de ces incidents : pour la plupart des appels reçus, j'ai été capable de déterminer l'origine du problème ou, à défaut de mettre en place un processus de diagnostic permettant de déterminer cette cause. Par la suite, j'ai également été capable de proposer puis de mettre en production des solutions de résolution ou de contournement pour ces problèmes.
\paragraph{}
L'exemple typique pour illustrer ces propos est le cas de la défaillance d'un équipement (chose qui arrive fréquemment). Dans ce cas, il faut prendre en charge l'incident le plus vite possible afin de rétablir le service au client. Grâce à l'expérience acquise au contact d'incidents de ce type et à mes connaissances du fonctionnement du réseau, je suis capable d'identifier rapidement l'origine de la panne : défaut de courant, défaut de câble, mise en sécurité d'un appareil, défaut de configuration... De plus dans chaque cas, je suis apte à mettre en place une solution de réparation efficace.
\paragraph{}
Une nouveauté pour moi cette année a été la gestion d'incidents avec l'aide de nos fournisseurs. En effet, en plus de l'achat des équipements, Vallourec loue un service de maintenance pour ces appareils chez un de nos fournisseurs. Ces contrats agissent comme une assurance contre les défaillances des appareils en permettant de demander un remplacement en cas de problème matériel et le prêt d'une expertise technique en cas de problème de configuration et/ou d'architecture. Lors de la gestion de certains incidents, j'ai donc été amené à prendre contact et gérer les échanges avec ces fournisseurs pour trouver une solution.\\
Ce nouvel aspect m'a permis de progresser sur plusieurs points : premièrement sur les échanges avec des employés d'entreprises extérieures et deuxièmement sur des notions techniques que je ne connaissais pas, expliqués par des experts en la matière.

\subsection{Remplacement d'équipements en fin de vie} % + d'autonomie
Cette partie fait suite au projet de remplacement des appareils en fin de vie auquel j'avais participé au début de 2015. Pour rappel : au bout de quelques années, les équipements réseau ne sont plus mis à jour par leur fabricant (Cisco dans le cas de Vallourec). Il peu alors arriver que les derniers logiciels contiennent des failles de sécurité qui puissent alors être exploitées à des fins malveillantes. Pour éviter cette éventualité, les appareils qui ne sont plus supportés (en fin de vie) doivent être remplacés.
\paragraph{}
Connaissant bien le sujet de ce projet, celui-ci a été plus intéressant pour son aspect organisationnel que pour son aspect technique. Pour la préparation de l'intervention, j'ai pu profiter d'une grande liberté pour configurer les appareils et l'organisation de l'intervention (préparation du déplacement, planification des étapes, établissement des procédures de test).
\paragraph{}
Cette tâche qui s'inclut dans un projet plus large mais qui fait quand même partie des tâches de maintenance a été une expérience intéressante  car elle m'a permis d'apprendre à gérer une intervention avec impact comme celle de façon à ce que tout se déroule correctement.

\subsection{Diagnostic pont Wifi}
Cette section fait écho au projet de mise en place d'un pont WiFi, détaillé dans la partie \ref{pont_wifi}. Peu de temps après la mise en production de cette infrastructure, les utilisateurs ont remonté à l'équipe des gros problèmes d'instabilité de la connexion qui fonctionnait très mal. Ayant réalisé les tests et la configuration du projet, j'ai naturellement pris en charge le problème.
\paragraph{}
Après de nombreuses recherches pour établir un diagnostic et assisté de mes collègues, nous n'avons pas pu trouver la cause du problème. Il a alors été décidé d'ouvrir un incident chez notre fournisseur (comme expliqué précédemment) afin d'obtenir une assistance technique. J'ai donc pris en charge les échanges avec les experts ainsi que la communication avec les utilisateurs en usine.
\paragraph{}
Après de nombreux échanges avec les experts techniques et des discussions sur la configuration des appareils, je me suis replongé dans le problème en établissant une procédure de diagnostic plus rigoureuse plutôt que de me reposer sur l'expertise du fournisseur. Après récupération des informations remontées par les les équipements et analyse approfondie en m'appuyant sur des documentations, je suis parvenu à identifier l'origine du problème. Les déconnexions étaient dues à des perturbations du signal WiFi par des ondes radar émises à proximité. J'ai donc pu déterminer comment mettre en place une solution : il fallait changer la fréquence de communication du pont WiFi pour éviter les perturbations. Après plusieurs tests en périodes creuses, la nouvelle la fréquence a été changée sur le pont qui s'est stabilisé et a permis aux utilisateurs de retrouver une connexion réseau fonctionnelle.
\paragraph{}
Le travail sur le diagnostic du pont WiFi m'a particulièrement intéressé. Au delà de la satisfaction d'être parvenu à résoudre le problème, j'ai appris que mettre en place une procédure de diagnostic rigoureuse permettait d'organiser les idées et d'éliminer des causes possibles pour déterminer la véritable cause. J'ignorais également que les ondes radar pouvaient perturber le fonctionnement du réseau WiFi.

\subsection{Recul sur le \textit{run}}
Avec le recul, je remarque que les tâches de maintenance opérationnelle on peu changé depuis l'année précédente et que celles-ci ont occupé la majorité de mon temps. Grâce à l'expérience acquise les années précédentes, j'ai pu être beaucoup plus efficace dans la prise en charge et la gestion d'incidents. Cet aspect du travail est très gratifiant, j'apprécie apporter mon aide à mes collègues et clients et entendre leur satisfaction.
\paragraph{}
En revanche, ayant fait de grands progrès auparavant, ces tâches m'ont de moins en moins intéressé au fil du temps. De plus, avec le ralentissement de l'activité des usines, le nombre de tâches de maintenance à réaliser a considérablement diminué, résultant en des périodes d'oisiveté particulièrement frustrantes.

\section{Sécurisation du réseau}
\subsection{Enjeux}
\subsection{Cisco ACS}
\subsection{Projet \textit{Common ACS}}
%Ajouter les présentations et exposés et schémas (annexe aussi)
\subsection{Projet 802.1X} %machines + iphones
\subsection{Résultats}%apports personnels et pour l'entreprise
\section{Supervision et gestion des équipements}
\subsection{Objectifs et enjeux}
\subsection{Étude de différents outils}
\subsection{Résultats}
\subsection{Prise en main des outils choisis}
\subsection{Rapports de performance}
\subsection{Gestion des configurations}
\section{Architectures WiFi}
% FAIRE DES SCHEMAS !!!
\subsection{Changement d'architecture}
\subsection{Pont WiFi contrôlé}\label{pont_wifi}
\section{Projets et missions transverses}
\subsection{Définition, objectifs et enjeux}
\subsection{Mise à jour des loadbalancers F5}
\subsection{Suppression des firewalls PaloAlto}
\subsection{Projet \textit{DataCenter revamping}}
\subsection{Mise à jour des appareils de visioconférence}
\subsection{Mise ne production d'un serveur Hyper-V}
\subsection{Résultats et enseignements}
\section{Retours et enseignements}

\chapter{Formation, compétences et métier}
\section{Vision du métier de l'ingénieur}
\section{Métier cible de la formation}
\section{Progression au cours de la formation}
\section{Compétences et capacités}
\section{Mise en perspectives et limites}
% lâche touuuut !
\section{Retour sur la formation par alternance}
\section{Liens avec les enseignements}

\chapter{International}
\section{Contexte de travail au quotidien}
\section{Écosse ??}

\chapter{Conclusion}

Je regrette que cette troisième année ait été trop ressemblante à la deuxième. Je m'attendrais à travailler sur des sujets de plus en plus complexes et intéressants en me basant sur les nombreuses connaissances acquises mais je me suis vu confier assez peu de sujets au final, en arrivant même à devoir trouver des tâches par moi même pour m'occuper.

\end{document}